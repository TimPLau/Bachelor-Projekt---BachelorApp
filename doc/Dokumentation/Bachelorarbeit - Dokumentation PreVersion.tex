\documentclass{scrreprt}
\usepackage[utf8]{inputenc}
\usepackage[T1]{fontenc}
\usepackage{lmodern}
\usepackage[ngerman]{babel}
\usepackage{amsmath}
\usepackage{hyperref}
\usepackage{booktabs}
\usepackage{subfigure} 
\usepackage{graphicx}
\usepackage{float}
\usepackage{multirow}
\usepackage{pdflscape}
%\usepackage{lscape}
\setlength{\parindent}{0em} 
\usepackage[citestyle=authoryear,natbib=true,style=alphabetic,backend=bibtex]{biblatex}
\usepackage{tabularx}

\newcommand*\adjust{\setlength\hsize{3\hsize+4\tabcolsep}} 

\renewcommand{\theenumi}{\arabic{enumi}}
\renewcommand{\labelenumi}{\textbf{PF\theenumi}}
\renewcommand{\theenumii}{\arabic{enumii}}
\renewcommand{\labelenumii}{\textbf{[Req\theenumi.\theenumii]}}
\renewcommand{\theenumiii}{\arabic{enumiii}}
\renewcommand{\labelenumiii}{[\theenumi.\theenumii.\theenumiii]}

\addbibresource{lit.bib} 

\title{Eine gamifizierte Howto-App für Bachelorarbeiten}
\author{Tim-Pascal Lau}
\date{28.05.2018}
\begin{document}
\maketitle
\tableofcontents
\listoftables
\listoffigures

\chapter{Einleitung}
Für Studierende im letzten Semester eines Bachelorstudiengangs, umfasst die wesentliche Prüfungsleistung das Verfassen einer Bachelorarbeit.
Die Auseinandersetzung mit komplexen Problemstellungen stellt jedoch erfahrungsgemäß für viele Studierende eine große Herausforderung dar, welche sich aus dem erstmaligem Zusammenspiel von selbständigem und eigenverantwortlichen Arbeiten, sowie Problemlösen mittels erworbener Fach- und Methodenkenntnisse über einen längeren (in etwa dreimonatigen) Zeitraum ergibt.

\section{Motivation}
Im Verlauf des Studiums sollten Studierende folgendes Wissen und folgende Fähigkeiten erworben haben und zielgerichtet Anwenden können:
\begin{itemize}
\item Studiengang-spezifisches Grundlagenwissen
\item Wissensansammlung über fachspezifische Methoden und deren Eigenschaften
\item Fähigkeit, komplexe Probleme zu erkennen, zu strukturieren und systematisch mittels geeigneter Methoden zu bearbeiten
\end{itemize}
Es kommt im Kontext von Bachelorarbeiten dennoch oftmals zu Schwierigkeiten, das erworbene Wissen und Fähigkeiten zielgerichtet auf reale Probleme anzuwenden und deren Ergebnisse zusammenhängend zu dokumentieren.

\section{Lösungsansatz}
Ein möglicher im Rahmen dieser Bachelorarbeit zu verfolgender Lösungsansatz wäre es, eine Software zu entwickeln, welche unterstützend und wegweisend bei dem systematischen Vorgehen bei komplexen Problemstellungen fungieren könnte und diese den Studierenden zugänglich zu machen.
Hierbei soll es nicht darum gehen, den Studierenden die eigentliche Arbeit abzunehmen, sondern vielmehr darum, Studierende hinsichtlich Vorgehen und Methodenauswahl zielgerichtet zu unterstützen.

\section{Zielsetzung}
Mit einem solchen Ansatz soll es Studierenden ermöglicht werden, ihr gelerntes Wissen durch Fokussierung bestimmter Aufgaben und Zusammenhänge im Rahmen ihres eigenen Bachelorprojekts auf die Realität zu übertragen und somit einen motivierenden, sowie gleichermaßen fordernden Rahmen zu schaffen, um ihr Bachelorprojekt erfolgreich abzuschließen.

\section{Aufgabenbeschreibung}
Im Rahmen dieser Bachelorarbeit soll hierfür eine mobile Applikation entwickelt werden, die den Studierenden während der Dauer der Bachelorarbeit kontinuierlich "begleitet". Dabei sollen Gamificationansätze realisiert werden, welche motivierend im bei der Bearbeitung und dem Vorgehen der eigenen Bachelorarbeit wirken sollen. Dies soll beispielhaft für den Studiengang Informatik/Softwareentwicklung erfolgen. Eine Erweiterbarkeit für andere Studiengänge ist hierbei jedoch konzeptionell vorzusehen
Primäre Funktionen der Software ist Studierenden bei den folgenden Aufgaben begleitend zu unterstützen und fortwährend zu motivieren "am Ball zu bleiben":
\begin{itemize}
\item Brainstorming (zur Unterstützung der Ideenfindung für Bachelorarbeiten)
\item Recherche und Literaturverwaltung
\item Gliederung (unterschiedlicher Kategorien von Bachelorarbeiten, zum Beispiel mittels bewehrter Templates)
\item Zeitplanung/Fortschrittsverfolgung, sowie Erinnerungs- und Benachrichtigungsfunktion 
\item Problem-orientierte Anforderungsanalyse und deren Dokumentation
\item Problem-orientierte Methoden- und Tool-/Frameworkselektion und deren Dokumentation
\item Methoden-spezifische Aufbereitung von Ergebnissen
\item Problem-orientierte Nachweisführung und deren Dokumentation
\end{itemize}
Die Applikation soll mittels Flutter für Android und iOS entwickelt werden. Dabei soll erhoben werden, inwiefern sich Flutter für die Entwicklung solcher Apps eignet(Lessons Learned). 
Die im Rahmen der Aufgabenbeschreibung entstandenen Anforderungen werden durch die folgenden Teilaufgaben spezifiziert:
\begin{itemize}
\item Detaillierte Anforderungsanalyse oben angegebener Funktionen. Hierbei sind Studenten und Professoren des Studiengangs Informatik/Softwaretechnik geeignet einzubeziehen und relevante Literatur (insbesondere zu Gamification und Methoden der Informatik und des Softwareengineering) zu berücksichtigen.
\item Architekturentwurf der Anwendung (Erweiterbarkeit für andere Studiengänge ist konzeptionell vorzusehen)
\item Implementierung der Anwendung
\item Die Funktonsfähigkeit der App soll mittels Softwaretests geeignet nachgewiesen werden.
\item Die Nutzbarkeit der App soll systematisch evaluiert werden. Hierbei sind Studenten und Professoren des Studiengangs Informatik/Softwaretechnik geeignet einzubeziehen.
\item Dokumentation der oben angegebenen Schritte inklusive Bewertung der Nutzbarkeit des Frameworks Flutter für solche Arten von Apps.
\end{itemize}

\section{Ausblick auf die Bachelorarbeit}
\par In der folgenden Dokumentation der Bachelorarbeit werden verschiedene aufeinander aufbauende Prozesse, Teilschritte und Ergebnisse der Softwareentwicklung dokumentiert sein. Hierbei liegt die Priorität vor allem bei dem Pflegen der Nachvollziehbarkeit der dargestellten Informationen durch aufeinander aufbauende Kapitel und dem reflektieren der eigenen Gedankengänge.

\subsection{Beschreibung der Kernkomponenten}
\par Der Nachvollziehbarkeit halber empfiehlt es sich, Grundkenntnisse über die Basiskomponenten, wie dem Flutter Framework zu besitzen. Sollte dies nicht der Fall sein, so lassen sich in Kapitel \ref{chap:grundlagenkapitel} die nötigen Informationen nachlesen.

\subsection{Untersuchung des Problembereiches}
\par Das Kapitel \ref{chap:problemanalyse} stellt in detaillierter Ausführung und Beschreibung die Prozesse der Anforderungsermittlung und deren Auswertung, sowie Definition dar und legt somit wichtige Grundlagen und Anforderungen an die Software fest. Weiterhin werden im Laufe des Kapitels Einblicke in Strategien und Gedankengänge ermöglicht, welche zusätzliche Anhaltspunkte für die Nachvollziehbarkeit der weiteren Kapitel beitragen können.

\subsection{Konzeptvortellung der Applikation}
\par \ref{chap:konzept}


\subsection{Architektur der Software}
\par Alle Informationen zur Softwarearchitektur, zu den Entwurfsentscheidungen, sowie der Berücksichtigung der Erweiterbarkeit der Software, werden im Kapitel \ref{chap:architektur}  behandelt.

\subsection{Implementierung}
Informationen zur detaillierten Implementierung der in der Aufgabenbeschreibung definierten Funktionen, sowie die Umsetzung der Benutzeroberfläche, werden im Kapitel \ref{chap:implementierung} behandelt. Unter Einbezug beispielhafter Codeauszüge werden hier die Funktionsweisen der Software aufgeführt und beschrieben.

\subsection{Validierung und Verifikation}
Die Nachweisführung der Softwareanforderungen, der Usability-Anforderungen, sowie die Auswertung der Nützlichkeit der Verwendung des Frameworks Flutter bei der Entwicklung dieser App lassen sich im Kapitel \ref{chap:nachweisführung} nachlesen.

\subsection{Präsentation der Ergebnisse}
Abschließend folgt im Kapitel \ref{chap:ergebnisse} eine Zusammenfassung der erreichten Ergebnisse und eine Reflexion der Teilschritte, sowie die Abschlussbetrachtung des gesamten Projekts.


\chapter{Beschreibung der Kernkomponenten} \label{chap:grundlagenkapitel}

\section{Vorstellung des Frameworks Flutter}
Flutter ist ein von Google entwickeltes opensource Framework, welches auf die Entwicklung mobiler 2D-Applikationen für Android- und iOS-Betriebssysteme ausgelegt ist. Beworben wird Flutter durch das Hervorheben der Einfachheit der Benutzung, die schnell zu erreichenden Fortschritte bei der Implementierung von Softwarefunktionen, sowie den Gestaltungsmöglichkeiten der Benutzeroberfläche und den hochqualitativen Ergebnissen.

\subsection{Flutter Systemarchitektur} 
Das Flutter Framework besteht aus drei verschiedenen Basiskomponenten, welche im folgenden Abschnitt kurz erläutert werden.
\begin{itemize}
\item{\textbf{Flutter Engine}}

Die C/C++ basierte Flutter Engine, setzt sich aus verschiedenen Kerntechnologien zusammen. Zum einen die open source 2D Graphics Libary Skia\citep{Skia1}, welche seit 2005 zu Google gehört und zum anderen die Dart Virtual Machine.
\item{\textbf{Foundation Libary}}

Die Foundation Libary, welche in Dart geschrieben wurde, stellt Basisklassen und -funktionen zur Verfügung und dient der Konstruktion von Applikationen mittels Flutter
\item{\textbf{Design-specific Widgets}}

Das Flutter Framework stellt zwei verschiedene Arten von Widgets zur Verfügung, welche zugehörig zu den jeweiligen Design Sprachen von Google Material Design\citep{Mat1}, welche 2014 entwickelt wurde und die iOS Design kopierende Design Sprache Cupertino\citep{Cup1}.
\end{itemize}

		...

\section{Firebase}

\section{Gamification} \label{sec:grundlagenkapitelGamification}
Vergleiche \citep{Strahringer2017}57

\subsection{Abgrenzung zu anderen Ansätzen}

\subsection{Bekannte Spiel-Design-Elemente}
Im folgenden Kapitel werden die wichtigsten Spiel-Design-Elemente \citep[vgl.]{blohm2013gamification} Vorgestellt und beschrieben. 
Vergleiche \citep[Kapitel 2.2.2 Analyse einzelner Spiel-Design-Elemente]{Sailer2016}\\

\begin{itemize}
\item Dokumentation von Verhaltensweisen
\par Durch die ständige Dokumentation von der eigenen Verhaltensweisen, werden Fortschritte für den Anwender sichtbar gemacht. Diese Fortschritte können ein Gefühl von hoher Leistungsfähigkeit auslösen 
\item Punkte 
\par Punktesysteme sind in ihrer Einfachheit eine weitverbreitete Form der Spiel-Design-Elemente, welche hauptsächlich eine Feedbackfunktion erfüllen. Diese Feedbackfunktion kann dem Spieler durch diverse weitere Möglichkeiten, wie zum Beispiel das Aufsteigen eine Levels, unterschiedlich stark verdeutlicht werden, weshalb Punkte häufig als Grundbestandteil von Gamification-Anwendungen auftreten. Weiterhin symbolisieren Punkte dem Spieler seinen derzeitigen Spielstand und bieten bei der Kombination mit Bestenlisten eine Vergleichsmöglichkeit zu anderen Spielern. In dieser Kombination können Punktesysteme also je nach Zielstellung auch Gewinner einer Herausforderung identifizieren.
\par In den zwei empirischen Untersuchungen \citep{mekler2013disassembling} und \citep{mekler2013points}, wurde nachgewiesen, dass Punkte durchaus einen leistungsfördernden Effekt auf Spieler haben können.
\item Ranglisten 
\item Ränge, Levels, Reputationspunkte 
\item Gruppenaufgaben 
\item Zeitdruck, Aufgaben, Missionen 
\item Avatare, virtuelle Welten, virtueller Handel
\end{itemize}

\begin{tabularx}{\textwidth}{l|l|X}
	\toprule
	\textbf{Spiel-Mechanik} & \textbf{Spiel-Dynamik} & \textbf{Motiv}\\ \midrule
	Dokumentation von Verhaltensweisen & Exploration & Wissbegierde\\
	Punktesysteme, Badges, Trophäen & Sammeln  & Leistung\\
	Ranglisten & Wettbewerb & Soziale Anerkennung\\ 
	Ränge, Levels, Reputationspunkte & Statuserwerb  & \\
	Gruppenaufgaben & Zusammenarbeit  & Sozialer Austausch\\
	Zeitdruck, Aufgaben, Missionen & Herausforderung & Kognitive Stimulation\\ 
	Avatare, virtuelle Welten, virtueller Handel & Entwickeln/Organisieren & Selbstbestimmung \\ 
	\bottomrule
\end{tabularx}
\captionof{table}{Übersicht der Spiel-Design-Elemente \citep[vgl.]{blohm2013gamification}}
\label{tab:spielDesignElemente}
\subsection{Nutzergruppen nach Bartle} \label{sub:nutzergruppenBartle}
\par Nach Bartle \citep{bartle1996hearts} werden vier verschiedene Nutzertypen voneinander unterschieden, welche im folgenden kurz erläutert werden.\\
Zudem werden zu den jeweils genannten Benutzergruppen die bekannten Spiel-Design Elemente ihrer Wirksamkeit bei der Steigerung der Motivation zugeordnet. (TODO)

\begin{itemize}
\item \textbf{Der Killer}
\par Die Killer legen den Fokus auf den Wettkampf und somit beruht ihre Motivationsquelle auf stark kompetitiv ausgeprägte Ursprünge. Sie fühlen sich dadurch motiviert, den Kontakt mit anderen Individuen im Rahmen eines Wettkampf zu überstehen und sich durch den eigenen Sieg gegenüber anderen Personen zu behaupten.

\item \textbf{Der Socialiser}
\par Die Gruppe der Socialiser sind immer auf der Suche nach sozialer Interaktivität. Ohne diese, können sie der Welt wenig bis keine Motivation abgewinnen. Die Spielwelt in der sie sich bewegen, ist vor allem nur der Mittel zum Zweck der Kommunikation. Der größte Stolz der Socialiser ist die Freundschaft und der Kontakt zu anderen Personen:

\item \textbf{Der Achiever}
\par Der Achiever ist im allgemeinen der Spielertyp, welcher sich auf das Erreichen von Leistungen konzentriert. Dieser wird so eingeschätzt, dass er die möglichst optimale geforderte Leistung erbringen will und er somit seine Motivation unter anderem auf das schrittweise Sammeln von Erfolgen und erfüllen von Aufgaben aufbaut.

\item \textbf{Der Explorer}
\par Die Explorer legen den Hauptfokus auf die Welt in der sie sich bewegen und möchten im Zusammenspiel mit dieser Spielwelt, durch immer wieder neu gestellte Aufgaben und erlebte Überraschungen alles entdecken, was es zu entdecken gibt. Sie sind sehr ambitioniert und empfinden Stolz für Ihr Wissen und Ihre Erfahrung in der Spielwelt. 
\par Die Explorer haben vor allem Interesse an der Spielwelt, und weniger an den Personen, die sie im Laufe der Erkundung treffen.
\end{itemize}

\begin{figure}[H]
	\centering
	\includegraphics[width=0.8\textwidth, keepaspectratio]{Bilder/Diagramme/InterestGraphBartle.png}
	\caption{Interest Graph nach Bartle \cite{bartle1996hearts}}
	\label{img:interetGraphBartle}
\end{figure}

\chapter{Untersuchung des Problembereiches} \label{chap:problemanalyse}
\section{Identifikation der Interessengruppen}
\par In diesem Kapitel werden die identifizierten Interessengruppen dokumentiert und in Form von Steckbriefen in den verschiedenen Kategorien \textbf{Einfluss}, \textbf{Einstellung}, \textbf{Erwartungen}, sowie \textbf{Bemerkungen} beschrieben.  
\subsection{Studierende}
\textbf{Einfluss:} Hoch\\\\
\textbf{Einstellung:} Positiv\\\\
\textbf{Erwartungen:}\par Optimales Ergebnis und weniger \glqq Fallstricke\grqq{} während der Bearbeitung der Bachelorarbeit, sowie ein allgemein \glqq nicht zu überfordernder\grqq{} Durchlauf der Bachelorarbeit durch weniger Unwissenheit vor/während der Bachelorarbeit\\\\
\textbf{Bemerkungen:}\par Hauptzielgruppe des Projekts/Spätere potenzielle Anwender der Applikation

\subsection{AStA}
\textbf{Einfluss:} Hoch\\\\
\textbf{Einstellung:} Positiv\\\\
\textbf{Erwartungen:}\par ... folgt\\\\
\textbf{Bemerkungen:}\par Stellt das Sprachrohr der Studierenden dar und vertritt somit deren Meinungen, Interessen und Ziele
\newpage

\subsection{Betreuer der eigenen Bachelorarbeit}
\textbf{Einfluss:} Hoch\\\\
\textbf{Einstellung:} Positiv\\\\
\textbf{Erwartungen:}\par Eine Steigende Bereitschaft/Motivation der Studierenden im Rahmen des Bachelor-Seminars Beiträge zu erbringen, sowie eine steigende Qualität der Kommunikation mit dem Betreuer und der damit zusammenhängenden Qualität der Bearbeitung der Bachelorarbeit und des Ergebnisses.
\\\\Detaillierte Angabe der Erwartungen:
\begin{itemize}
\item Zielgerichteter(er) Methoden-Einsatz von allen Methoden, die im Informatik/SWT-Studium gelehrt werden
\item Zielgerichtete(re) Vorbereitung auf Besprechungen mit dem Betreuer
\item Bessere Lesbarkeit von Abschlussarbeiten
\item Bessere \glqq rote Fäden\grqq{} in Abschlussarbeiten
\item Bessere Zeitplanung
\item Systematische(re) Problemanalyse und Anforderungserhebung und deren Dokumentation
\item Bessere Architekturentwicklung und deren Dokumentation
\item Systematische(re) Nachweisführung und deren Dokumentation
\end{itemize}
\textbf{Bemerkungen:}\par Führt das Bachelorseminar und hat somit direkten Kontakt mit der Zielgruppe, trägt wichtige Erfahrungswerte mit sich, woran es üblicherweise bei der Fertigstellung von Bachelorarbeiten mangelt
\newpage

\subsection{Professoren}
\textbf{Einfluss:} Hoch\\\\
\textbf{Einstellung:} Positiv\\\\
\textbf{Erwartungen:}\par ... folgt\\\\
\textbf{Bemerkungen:}\par Tragen Erfahrungswerte durch das Betreuen und Bewerten von Bachelorarbeiten, kennt die Probleme der Studenten und hat detaillierten Einblick in die Schwierigkeiten der Zielgruppe

\subsection{Präsidium}
\textbf{Einfluss:} Gering\\\\
\textbf{Einstellung:} Positiv\\\\
\textbf{Erwartungen:}
\begin{itemize}
\item Wenn Applikation positiven Einfluss auf die Ergebnisse von Abschlussarbeiten, kann die Fachhochschule Lübeck ihren Ruf festigen werden
\item Erhöhung der Wettbewerbsfähigkeit durch bessere Leistung/bessere Abschlüsse der Studierenden
\end{itemize}
\textbf{Bemerkungen:}\par Machtpromotor
\newpage


\section{Hypothese}
\par Die Auseinandersetzung mit komplexen Problemstellungen, wie die als abschließende Prüfungsleistung des Studiums zu erarbeitende Bachelorarbeit, stellt erfahrungsgemäß für viele Studierende eine große Herausforderung dar. Diese Herausforderung ergibt sich aus dem erstmaligen Zusammenspiel von selbstständigem und eigenverantwortlichem Arbeiten, sowie dem Problemlösen mittels erworbener Fach- und Methodenkenntnisse über einen längeren (in etwa dreimonatigen) Zeitraum.
Trotz des Verlaufs des Studiums, des angeeigneten Wissens und der somit zahlreich erworbenen Fähigkeiten, kommt es im Kontext von Bachelorarbeiten dennoch oftmals zu Schwierigkeiten, diesen Zusammenhang auf reale Probleme abzubilden und zu dokumentieren.

\section{Wahl der Analysestrategie}
\par Um den Problembereich zu ermitteln und somit eine Analysegrundlage zu erschaffen, werden Personengruppen der Fachhochschule Lübeck, durch verschiedene Befragungs- und Analysemethoden in das Projekt miteinbezogen. Dies soll einen detaillierten Einblick in die Sichtweisen der unterschiedlich beteiligten Personen und Interessengruppen ermöglichen und somit eine Grundlage für das Verständnis der aktuellen Situation bilden.

\par Als primäre Einflussgeber wurden in diesem Rahmen die Gruppe der Professoren, sowie die Gruppe der Studierenden identifiziert. Diese Entscheidung wurde aufgrund der im Rahmen einer Bachelorarbeit  existierenden unterschiedlichen Sichtweisen, sowie Erfahrungsständen von Betreuern und Bacheloranden getätigt.

\begin{figure}[H]
	\centering
	\includegraphics[width=0.95\textwidth, keepaspectratio]{Bilder/Diagramme/Analysestrategie.png}
	\caption{Beschreibung der Analysestrategie}
	\label{img:analysestrategie}
\end{figure}

\par Durch die Gespräche mit den Studierenden ließen sich Erkenntnisse über den Umfang des Problems gewinnen. Auf Grundlage dieser Erkenntnisse wurde die Analysestrategie verfeinert und für die Umsetzung vorbereitet (siehe Abbildung \ref{img:analysestrategie}), welche im folgenden Verlauf beschrieben werden.
\par Die Interviews sollen als Grundlage zur Konzeption und Design weiterer Befragungsstrategien dienen und stellen in diesem Umfang einen Ausgangspunkt für die Problemanalyse dar. Weiterhin werden die gewählten Interviewpartner bei Zustimmung, über das gesamte Projekt und nach Absprache, auch in die Evaluation der Applikation miteinbezogen.
\par\medskip Durch das Durchführen von Gruppeninterviews, sollen die Studierenden zu Diskussionen angeregt sein, welche von dem Leiter des Interviews durchaus auch motiviert werden können. Dies hat den Zweck, die Situation und die Rolle der Studierenden für den Interviewer erkenntlich zu machen.  Diese persönlichen Interviews dienen in erster Linie also nicht zur Erhebung von Daten, sondern als strategischer Orientierungspunkt für mich, den Projektleiter.\\

\subsection{Einbezug der Professoren}
\par Für die Interessengruppe der Professoren aus dem Fachbereich Informatik, ist als Grundlage der Datenerhebung einerseits das Durchführen von Einzelinterviews mit einer ausgewählten Gruppe von Professoren vorgesehen, während eine weitere Gruppe von Professoren schriftlich per E-Mail befragt wird (siehe Abbildung \ref{img:analysestrategie}). Dies bietet sowohl den Zugriff auf die unmittelbaren Erfahrungen der einzelnen Professoren als Spezialisten in den jeweiligen Fachgebieten, als auch auf die Erfahrungen der Professoren in der Position eines Betreuers und Ansprechpartners für Bacheloranden. 
\par Durch das Durchführen von Einzelinterviews wird ermöglicht, die Erwartungen seitens der Professoren an die Bacheloranden im Detail zu identifizieren und die, in dieser Hinsicht priorisierten inhaltlichen und methodischen Aspekte bei der Bearbeitung einer Bachelorarbeit herauszuarbeiten. Die Aufteilung auf Einzelinterviews und E-Mail Befragungen bietet den Vorteil, beide Strategien simultan zu verfolgen und nach Abschluss der Datenerhebung sowohl die detaillierten Einzelinterviews, als auch die oberflächlicher ausfallenden E-Mail Antworten in bereits dokumentierter Form vorliegen zu haben, um diese dann auszuwerten.\\

\subsection{Einbezug der Studierenden}
\par Nach Absprache mit den intervieweten Studierenden, werden diese weiterhin per schriftlicher Befragung in die Entwicklung des Projektes, unter ergänzender Zunahme anderer Studierender, eingebunden.
\par In diesem Umfang werden die teilnehmenden Studierenden mittels Bildern, Fragen und Gestaltungsmöglichkeiten der Applikation regelmäßig über den aktuellen Stand der Entwicklung informiert. Auf diese Weise sollte es möglich sein, ein breites, jedoch persönliches Feedback zu erhalten, da sich die Einbindung der Studierenden auf diese Weise sehr gut automatisieren lässt. Wesentlich dabei ist auch die Hoffnung, dass die Reaktionsfreudigkeit der Studierenden höher ausfällt, als für die im Vergleich existierende Alternative der zeitaufwändigeren persönlichen Interviews.
\par Durch die Durchführung der schriftlichen Befragung wird der Ablauf bereits dokumentiert und der Fragesteller, sowie die Studierenden haben jederzeit die Möglichkeit Fragen und Anregungen zu teilen.
\par Miteinbezogen werden vorzugsweise alle Studierende der oberen Semester, unabhängig davon, ob sie sich vor Beginn, während der Bearbeitung oder nach Abschluss der Bachelorarbeit befinden, da die unterschiedlichen Sichtweisen und Erfahrungsstände wichtige Impulse für die Entwicklung der Applikation geben könnten.
\par\medskip Des weiteren wird eine Online-Umfrage den Teil der Datenerhebung darstellen, der quantitative Ergebnisse erzielen soll und deren Aussage somit eine nicht durch die schriftlichen Befragungen abgebildete Menge darstellt.
\par Die Umfrage richtet sich nicht nur an die Studierenden des Studiengangs Informatik/Softwaretechnik, sondern im Sinne der Erweiterbarkeit für andere Studiengänge, an Studierende im Allgemeinen, welche sich ab dem 4. Semester bereits in der zweiten Hälfte des Studiums befinden und somit schon einen ein­schätz­baren Erfahrungsschatz aufgebaut haben.

\section{Ergebnisse der Professoren}
\par Im Folgenden sind die gewonnenen Eindrücke und Kenntnisse der Einzelinterviews mit den Professoren des Fachbereichs Informatik durch Themenkategorien geordnet und in zusammengefasster Form dokumentiert.
\par Es haben insgesamt sechs Professoren an persönlichen Einzelinterviews teilgenommen, wobei es für fünf Interviews gestattet wurde, eine Tonaufzeichnung anzufertigen.
\par\medskip Des Weiteren wird auch eine schriftliche Befragungen miteinbezogen, die jedoch das gleiche Fragespektrum wie die Interviews einnimmt.
\par\bigskip \textbf{Allgemeine Informationen}
\par Der zeitliche Rahmen der fünf aufgezeichneten Interviews erstreckt sich über einen Zeitrum von etwa 30 bis 45 Minuten. Für die Auswertung der aufgezeichneten Interviews wurden die Aussagen der Interviewpartner, auf Grundlage der vorliegenden Audioaufnahmen, unter Berücksichtigung des Kontextes aufbereitet und werden nachfolgend dargestellt. Ein weiteres Einzelinterview, welches nicht aufgezeichnet wurde, erstreckte sich über einen Zeitraum von 75 Minuten. Für dieses Interview wurden lediglich begleitende Feldnotizen angefertigt. Diese Feldnotizen wurden im Anschluss des Interviews aufbereitet und fließen zusammen mit den Ergebnissen der schriftlichen Befragung in die folgende Beschreibung ein.
\par\medskip Die gewählten Kategorien ergeben sich aus dem gewählten Auswertungsverfahren der qualitativen Inhaltsanalyse nach Mayring\citep{Mayring2015} und basieren in diesem Kontext auf die gemäß des Verfahrens herausgearbeiteten Codings.

\newpage
\subsection{Art der Arbeit}
\par Im Laufe der Interviews wurden verschiedene Typen von Arbeiten versucht zu identifizieren. Dabei geht es vor allem darum, die Vielfalt der typischen Arbeiten des Studiengangs Informatik/Softwareentwicklung zu erfassen und somit einen Überblick über die Situation zu bekommen.\\
Als im allgemeinen auftretenden Arten der Arbeit wurden die Klassen \textbf{Entwickelnde Arbeit}  und \textbf{Evaluierende Arbeit} identifiziert. Weiterhin gibt es auch \textbf{reine Literaturarbeiten}, welche in dem Studiengang Informatik/Softwareentwicklung jedoch nicht oder nur in einem sehr geringen Vorkommen auftreten.

\par\medskip Es folgt eine stichpunktartige Ausführung der gewonnenen Erkenntnisse:

\begin{itemize}
\item \textbf{Konstruktiv/Entwickelnd - Durchlauf des Softwareentwicklungszyklus}
	\begin{itemize}
	\item \textbf{Anforderungsanalyse}
	\par Unterschiedlich komplex, je nach individueller Aufgabenstellung und 
	Rahmenbedingungen. Ausschlaggebend hierfür ist vor allem, ob es sich um eine interne 
	Arbeit an der Fachhochschule oder eine externe Arbeit in einem 	
	Unternehmensumfeld handelt, bei der gegebenenfalls die Anforderungen schon definiert sind. 	
	Sollten bereits Anforderungen existieren, so ist das Infragestellen dieser Anforderungen 
	häufig Bestandteil der Aufgabe. 
	\item \textbf{Entwurf einer Softwarearchitektur}
	\par Je nach individueller Aufgabenstellung und Rahmenbedingungen können auch hier große 
	Unterschiede in der Bearbeitung liegen. Ein wichtiger Orientierungspunkt hierbei ist das 
	Vorhandensein von schon existierenden Softwareprodukten, welche entweder erweitert oder 
	ersetzt	werden könnten. Dies ist häufig bei externen Bachelorarbeiten zu erwarten. 
	Eine weitere Möglichkeit ist, dass es keine bereits vorhandene Softwarelösung gibt, 
	sondern diese von Grund auf entwickelt werden soll.
	\item \textbf{Implementierung und Evaluation eines Softwareprototyps}
	\par Es wird betont, dass es zumindest bei internen Arbeiten nicht unbedingt darum geht, 
	nach Abschluss der Bearbeitung der Bachelorarbeit, ein Softwareprodukt vorliegen zu haben, 
	was für die Markteinführung geeignet ist. Es wird deshalb oftmals von einer prototypischen 
	Implementierung gesprochen. Bei externen Arbeiten kann dies jedoch, im Sinne der Unternehmen 
	oder Organisationen, durchaus den Zielanforderungen entsprechen. Im Rahmen der Evaluation 
	gibt es eine Vielzahl von Möglichkeiten, die je nach Schwerpunkt der Arbeit den Fokus auf 
	verschiedene Ziele legt. 
	\par Softwaretesting findet in den entwickelnden Arbeiten bei 
	Implementierung grundsätzlich statt, jedoch gibt es eine Vielzahl an weiteren Aspekten, wie 
	Usability-Tests und Evaluation der Nützlichkeit einer Software, welche je nach 
	Themenschwerpunkt untersucht werden können.
	\item \textbf{Sonstige Anmerkungen}
	\par Typische Aufgabenstellungen könnten sein:
	\begin{itemize}
		\item[1] Entwicklung einer mobilen Applikation zur Interpretation von Bildmaterial.
		\item[2] Entwicklung einer mobilen Applikation zur Steigerung der Bereitschaft bei 
		Senioren und Seniorinnen, Fitnessaktivitäten auszuführen unter Einbezug von 
		Gamificationelementen.
		\item[3] Entwicklung einer Software zur Optimierung der täglichen Arbeitsabläufe in 
		Unternehmen A.
	\end{itemize}
	
	\par Ein weiterer wichtiger Aspekt sind die möglichen Interessenunterschiede zwischen dem 
	externen Unternehmen und dem internen Betreuer der Fachhochschule, welche einen 
	Einfluss auf die Inhalte der Bachelorarbeit haben können. Externe Unternehmen sind 
	tendenziell eher an dem resultierenden Ergebnis interessiert, während die internen Betreuer 
	darüber hinausgehend einen hohen Wert auf nachvollziehbare Methodik, Herangehensweise, 
	sowie Planung und dem sauberen wissenschaftlichen Arbeiten legen und somit ein  hohes 
	Interesse an dem Gesamtprozess haben.
	\end{itemize}
\item \textbf{Analytisch/Evaluierend - Vergleich, Auswertung und/oder Nachweis eines Aufgabengegenstandes}
	\begin{itemize}
	\item \textbf{Erstellen eines Kriterienkatalogs}
	\par Messbare Kriterien stellen die Grundlage des Experiments/der Auswertung dar und werden 
	meistens bereits zu Beginn, in Form eines Kriterienkatalogs festgelegt. 
	Ein wichtiger Aspekt ist herbei vor allem die Frage, wie die Kriterien gemessenen werden 
	können. Die damit zusammenhängende Aussagekraft der Kriterien soll hierbei hinterfragt und 
	diskutiert werden.
	\item \textbf{Aufbau des Experiments}
	\par Der Aufbau des Experiments hängt stark von dem Anwendungsfall, der Zielstellung und des 
	Untersuchungsgegenstandes ab. Hierbei werden die gewählten Strategien und Rahmenbedingungen 
	zusammenhängend erläutert und beschrieben, um ein nachvollziehbares Fundament für die 
	Durchführung des Experiments zu erschaffen, Abhängigkeiten darzustellen und Besonderheiten 
	zu klären. 
	\par Im Rahmen der Untersuchung werden beispielsweise Datenerhebungsmethoden wie 
	Online-Umfragen und Interviews geführt. 
	\par Sollten Vergleiche verschiedener Technologien 
	Gegenstand der Arbeit sein, so werden zum Beispiel auch Fallstudien durchgeführt. 
	\par Bei Auswertung vorhandener Technologien sind oftmals auch Machbarkeitsstudien zentraler 
	Bestandteil der Arbeit.
	\item \textbf{Durchführung des Experiments}
	\par Je nach Ausrichtung der Aufgabenstellung und des Themengebietes können hier 
	unterschiedliche Ansätze ausgeprägt und beschrieben sein, welche zuvor im Aufbau des 
	Experiments dargelegt wurden. 
	\item \textbf{Evaluation und Ergebnisauswertung}
	\par Die Evaluation der Ergebnisse und die damit zusammenhängende Diskussion ist der 
	zentrale Bestandteil der Arbeit. Alle vorherig getätigten Entscheidungen und Strategien 
	werden nun zusammenhängend mit der Problemstellung ausgewertet und weiterhin diskutiert. 
	\item \textbf{Sonstige Anmerkungen}
	\par Typische Aufgabenstellungen könnten sein:
	\begin{itemize}
		\item[1] Evaluation der Gesichtserkennungsdienste von Unternehmen A, Unternehmen B und 
		Unternehmen C.
		\item[2] Untersuchung des Verhaltens einer neuen Technologie A, im Vergleich mit einer 
		alten Technologie B.
		\item[3] Datenbankanalyse unter Anwendung von Machine-Learning-Alrogithmen
	\end{itemize}
	\end{itemize}
\item \textbf{Reine Literaturarbeiten}
\par Reine Recherchierende Arbeiten finden in dem Studiengang Informatik/Softwareentwicklung aufgrund dem geringen Interesse seitens der Studierenden kaum statt und werden aus Gründen der Vollständigkeit lediglich erwähnt und nicht ausgeführt.
\end{itemize}

\subsection{Erwartungen an den Bacheloranden}
\par Im Laufe der Interviews wurden die Professoren hinsichtlich Ihrer Erwartungen an die Bacheloranden befragt und haben in diesem Rahmen häufig gleiche oder ähnliche Punkte ausgeführt. Aus diesem Grund werden im folgenden Verlauf die Meinungen der befragten Professoren aus der Sicht als Betreuer, unter den jeweiligen Aspekten als zusammengefasstes Meinungsbild wiedergegeben.

\par\medskip Es folgt die Ausführung der gewonnenen Erkenntnisse:
\begin{itemize}
\item \textbf{Selbständiges Arbeiten}
\par Das selbstständige Arbeiten und Vorgehen ist eines der am häufigsten genannten Erwartungen, welches sich in unterschiedlichen Punkten zum Ausdruck bringen lässt. Dazu zählt vor allem das selbständige kommunizieren von Ergebnissen und das Einholen von Feedback, sowie die Transparenz bei Problemen oder Schwierigkeiten, um sich Hilfe von dem Betreuer zu holen. Es wurde mehrfach betont, dass es im Allgemeinen nicht die Aufgabe des Betreuers ist, nachzufragen und aufzufordern. Das Einbinden des Studierenden in die Erarbeitung der Aufgabenstellung ist ein häufig genannter Punkt, welcher bereits frühzeitig Engagement des Studierenden erfordert.
\par Des Weiteren stellt das Selbstständige Einarbeiten in die Probleme, die Auswahl geeigneter Methoden und Werkzeuge ein wesentlicher Inhalt der Arbeit dar.
\item \textbf{Die Vorgehensweise}
\par Sehr häufig wird betont, dass die Vorgehensweise hinsichtlich der wissenschaftlichen Arbeitsweise von essenzieller Bedeutung ist. In diesem Rahmen soll der Studierende auch zeigen, dass er in der Lage ist große Probleme systematisch in Teilprobleme zu zerlegen und diese unter Berücksichtigung der im Studium vermittelten Methoden, Modelle und Techniken zu bearbeiten. Sehr wichtig ist dabei das vorausschauende Planen von Teilprozessen, wie beispielsweise das Evaluieren der Ergebnisse. Dies sollte von Anfang an berücksichtigt werden und die in diesem Rahmen getroffenen Entscheidungen sollten nachvollziehbar erklärt werden können. Dazu zählt weiterhin das Erstellen eines Zeitplans, welcher sich über die Zeit jedoch durchaus verändern kann. Es wird sehr viel Wert darauf gelegt, zu sehen, dass die Studierenden einen weiten Blick auf das gesamte Projekt entwickeln und pflegen.
\item \textbf{Der Literaturteil}
\par Es wird betont, dass vor allem Wert auf einschlägige Quellen Wert gelegt wird. In diesem Umfang ist es wichtig, dass die Studierenden Literaturquellen verwenden sollten, die bereits eine anerkannte längere Gültigkeit besitzen. Weiterhin sollten die Studierenden über den Umfang von Grundlagenliteratur hinaus blicken und je nach Themengebiet und Arbeitsstand spezifischere Fachliteratur in die Arbeit einbeziehen. Dies kann auch bedeuten, dass auf wissenschaftliche Papiere und Primärquellen verwiesen werden soll. Je nach Thema und Aufgabenstellung kann der Literaturteil mehr oder weniger Umfangreich ausfallen, welches sich mit dem praktischen Teil der Arbeit ausgleichen kann.
\item \textbf{Der praktische Teil}
\par Im Allgemeinen soll der praktische Teil den Umfang der Aufgabenstellung abdecken und gegebene und/oder erhobene Anforderungen erfüllen. Je nach Thema und Aufgabenstellung der Arbeit nimmt dieser Teil einen höheren oder niedriger ausfallenden Umfang ein.
Der Studierende soll bei der praktischen Bearbeitung der Aufgabe zeigen, das er in der Lage ist, das im Studium gelernte Wissen, die kennengelernten Methoden und deren Ausführung umzusetzen. 
\item \textbf{Das Ergebnis der Bachelorarbeit}
\par Grundsätzlich soll die Bachelorarbeit aus zwei Teilen von Leistungen bestehen. Die von dem Studenten durchgeführte Literaturarbeit nimmt einen Teil der Arbeit ein, während die eigenständige praktische Leistung den anderen Teil erfüllt. Der Umfang der jeweiligen Anteile kann dabei je nach Themengebiet und Aufgabenstellung stark variieren.
\par Das Ergebnis der Bachelorarbeit, welches sich je nach Art der Arbeit voneinander stark unterscheiden kann, soll wissenschaftlich erarbeitet und somit nachvollziehbar und belegbar sein, sowie im Optimalfall die Aufgabenstellung erfüllen. Es kann jedoch auch vorkommen, dass das angestrebte Ziel aus verschiedenen Gründen nicht erreicht wurde. Dies muss nicht bedeuten, dass es zu einer schlechten Bewertung der Bachelorarbeit kommt, sofern der Grund oder die Erkenntnis über ein bestimmtes aufgetretenes Problems belegbar und nachvollziehbar dokumentiert ist.
\end{itemize}

\subsection{Häufig auftretende Probleme}
\par Es stellte sich im Verlauf des Interviews heraus, dass unterschiedliche Studierende immer wieder mit gleichen oder ähnlichen Problemen zu kämpfen haben. Im folgenden Verlauf werden diese genannten Probleme in aufbereiteter Form stichpunktartig beschrieben.

\par\medskip Es folgt die Ausführung der gewonnenen Erkenntnisse:
\begin{itemize}
\item \textbf{Zeitmanagement}
\par Als am häufigster genannter Aspekt ist das mangelhafte Zeitmanagement der Studierenden. Im Laufe der Bearbeitung der Bachelorarbeit kommt es häufig zu der Unterschätzung des nötigen  Zeitaufwandes, besonders hinsichtlich des Schreibens der Dokumentation. 
\par Viele Studierende schieben das Schreiben der Dokumentation auf einen späteren Zeitpunkt und geraten im späteren Verlauf der Bearbeitungszeit somit unter Zeitdruck. Die Betreuer haben unter diesen Umständen wenig Möglichkeiten, rechtzeitiges und hilfreiches Feedback zu liefern. Es wird häufig empfohlen, frühzeitig mit dem Schreiben anzufangen und dies begleitend zum Arbeitsfortschritt die Dokumentation an mehreren Stellen wachsen zu lassen. Trotz vieler Hinweise seitens der Betreuer, kommt es in diesen Belangen häufig zu starken Problemen.
Das Problem des mangelnden Zeitmanagements äußert sich unter anderem auch darin, dass die Studierenden sich bei der Bearbeitung in Details verlieren, da sie die Schwerpunkte der Arbeit nicht erkennen.
\item \textbf{Wissenschaftliches Arbeiten}
\par Die Studierenden erfassen teilweise nicht die Bedeutung des wissenschaftlichen Arbeiten. Es kommt immer wieder zu Schwierigkeiten und Unklarheiten über den eigentlichen Umfang der Arbeit und wodurch sich das wissenschaftliche Arbeiten auszeichnet. Oft wird der falsche Umgang mit Literatur und Quellen als negatives Beispiel genannt.
\par Ein Kritikpunkt ist, dass es gibt in dem Studiengang Informatik/Softwareentwicklung keinen Kurs gibt, welcher die Studienenden auf das wissenschaftliche Arbeiten vorbereitet. In einigen Wahlpflichtmodulen werden diesbezüglich zwar Ansätze im Rahmen von Projekten integriert, jedoch gilt dies somit nur für die an dem Wahlpflichtmodul teilnehmenden Studierenden und steht auch nicht im Fokus der Projektarbeit.
\item \textbf{Kommunikation}
\par Kommunikation und Transparenz ist ein weiteres angesprochenes Problem. Es kommt vor, dass Studierende und Betreuer unterschiedliche Ansichten über die Zusammenarbeit haben, welche sich dadurch äußern, dass die Studierenden auf Forderungen bezüglich Leistungen oder Ergebnissen der Betreuer warten oder aus sich sogar aus diversen Gründen nicht trauen, ihren aktuellen Arbeitsstand oder ihre Probleme mit dem Betreuer zu teilen. 
\item \textbf{Die Vorbereitung der Studierenden}
\par Bei der Vorbereitung der Studierenden nennen die beteiligten Interviewpartner unterschiedliche Aspekte, welche zum einen das mangelhafte selbständige Informieren der Studierenden kritisiert, andererseits jedoch auch eine optimaler zu gestaltende Vorbereitung der Studierenden seitens der Fachhochschule für den Studiengang Informatik/Softwareentwicklung.
\par Das angebotenen Bachelorarbeit Seminar wird positiv erwähnt, da es einen positiven Einfluss auf die Arbeit der Studienenden hat. Es müssen weniger Aspekte einer Bachelorarbeit erklärt werden, jedoch müssen viele Aspekte mehrfach wiederholt werden. 
\par Es wird betont, dass auch viele Informationsmaterialien im Lernraum der Fachhochschule Lübeck zu finden sind, auf die auch oft hingewiesen wird, jedoch von den Studierenden nicht in dem Umfang beachtet werden, für den die Materialien gedacht sind. Dabei wird unter anderem auch kritisiert, das die Informationsmaterialien teilweise schwer auffindbar sind, da sie nicht an einer zentralen Stelle, sondern verteilt im Lernraum liegen. 
\par Weiterhin wird jedoch auch betont, dass die Studierenden zu wenig Engagement aufbringen, sich trotz vieler Möglichkeiten selbstständig zu informieren. 
\end{itemize}

\subsection{Die Applikation - Wünsche, Chancen und Risiken}
\par In jedem Interview bekamen die Professoren abschließend die Möglichkeit, ihre Erwartungen an eine solche Applikationen auszuführen und besonders auf die, aus ihrer Sicht mögliche Risiken und Chancen einzugehen und diesbezüglich auch Anmerkungen oder Empfehlungen zu tätigen. Diese Anmerkungen werden im folgenden Verlauf zusammengefasst dargestellt.

\par\medskip Es folgt die Ausführung der gewonnenen Erkenntnisse:
\begin{itemize}
\item \textbf{Anregungen und Wünsche}
\begin{itemize}
\item \textbf{Neuer Kanal zu den Studierenden}
\par Es wird der Wunsch geäußert, dass die Applikation verwendet werden kann, um in zentraler Form konkrete interne oder externe Bachelorarbeit-Themen, sowie Beispielthemen angeben zu können, da dies an der Fachhochschule Lübeck bisher nicht ermöglicht ist.
\item \textbf{Plattform als Informationsquelle}
\par Es besteht der Wunsch, die im Lernraum vorliegenden Informationsmaterialien, durchaus auch in aufbereiteter Form, durch die Applikation den Studienreden zugänglicher zu machen.
\item \textbf{Applikation zur Unterstützung des Zeitmanagements}
\par Es wird der Wunsch geäußert, die Studierenden bei dem Zeitmanagement, unter anderem durch Erinnerungen, zu unterstützen. In diesem Rahmen wird der Vorschlag eingebracht, möglichst detaillierte Beschreibungen von Arbeitspaketen in dem Tool zu verlangen, damit die Studierenden dazu gezwungen sind, sich rechtzeitig mit der Aufwandseinschätzung zu beschäftigen.
\end{itemize}
\item \textbf{Chancen}
\begin{itemize}
\item Die Applikation als neuer Kanal für die Studierenden, der dafür dienen kann, dass Studierende sich besser informieren können. Dies wird besonders in Bezug auf die Formalien einer Bachelorarbeit betont, da viele Studierende gar nicht wissen was die Rahmenbedingungen einer Bachelorarbeit sind oder welche Regeln und Anforderungen es überhaupt gibt.
\item Die Applikation kann im Gegensatz zum Bachelorseminar begleitend zu der eigenen Bachelorarbeit genutzt werden kann. Dies kann dafür sorgen, dass die Aufnahmebereitschaft der Studierenden für Tipps, Empfehlungen und weiteren Aspekten gesteigert wird, da sie sich zu diesem Zeitpunkt mit dem Problem konfrontiert sehen und somit der Lerneffekt am höchsten ist. In diesem Ansatz wird auch die Chance erkannt, dass der Fokus der Studierenden zum richtigen Zeitpunkt auf bestimmte wichtige Fragen gelenkt werden können und somit grobe Fehler minimiert werden können.
\item Besseres Zeitmanagement der Studierenden und die somit geförderten organisatorischen Fähigkeiten der Studenten.
\item Die App könnte Probleme im Projektmanagement und bei der Gestaltung der Dokumentation minimieren.
\item Das Senken des Beratungsaufwandes für Professoren und somit das Minimieren von sich wiederholenden Arbeitsabläufen für die unterschiedlichen Bacheloranden wird als Chance genannt, da in einfacher Form auf eine Applikation verwiesen werden kann, die alle nötigen Informationen enthält.
\end{itemize}

\item \textbf{Risiken}
\begin{itemize}
\item Die Applikation kennt nicht den realen Stand der Bachelorarbeit, sondern die Studierenden sind für die Verwaltung selbst zuständig. Wenn der Benutzer eine der Aufgaben abhakt, stellt dies unter Umständen nicht den echten Zustand der Bachelorarbeit dar und könnte dem Studierenden einen falschen Eindruck des Fortschritts geben.
\item Die Applikation regt dazu an, sich durch das Zeitmanagementtool zu über-planen, was dafür sorgt, dass der Benutzer von der eigentlichen Arbeit abgehalten wird. In diesem Rahmen kann die Applikation dem Benutzer nicht die Eigenverantwortung abnehmen. Der Studierende kann der Applikation nicht die Schuld für einen Misserfolg geben.
\item Gamificationelemente könnten unter Umständen einen sehr begrenzten Effekt haben, da sie kein Interesse an einem Thema wecken können, sondern das Grundinteresse aus der Eigenmotivation erzeugt werden muss.
\item Befürchtung, dass Studenten gegebenenfalls die Applikation als Leitfaden als unumstößlich ansehen könnten und somit durch unterschiedliche Ansichten in einen Konflikt mit dem Betreuer geraten können.
\item Die App könnte missverstanden werden als Ersatz für die persönliche Betreuung – insbesondere fachliche Aspekte wird eine App naturgemäß nicht abdecken können. Es könnte weiterhin zu einem „Device Mismatch“ kommen, da Bachelorarbeiten üblicherweise nicht an mobilen Endgeräten entstehen man müsste also immer zwei Geräte bedienen: Notebook/Desktop-PC und Smartphone.

\end{itemize}

\end{itemize}

\newpage
\section{Ergebnisse der Studierenden}
\par Das folgende Kapitel liefert einen Einblick in die Ergebnisse der Befragung der Studierenden und beschreibt die in diesem Rahmen gewonnenen Erkenntnisse.
\par Im Fokus der Befragung stehen die Erwartungen an die Bachelorarbeit und den Betreuer, sowie die typischen Probleme und Sorgen der Studierenden. Die Erkenntnisse der Befragung werden abschnittsweise in zusammengefasster Form dargestellt. Die gesamten schriftlichen Befragungen sind im Anhang zu finden. (TODO) 

\subsection{Ergebnisse der schriftlichen Befragung der Studierenden} \label{sub:studentenchriftlichErgebnisse}
\par Informationen zu der schriftlichen Befragung:
\begin{itemize}
\item Insgesamt wurden 17 Studenten schriftlich befragt.
\end{itemize}

\par \textbf{Schriftliche Befragung der Studierenden}
\par Die erste Befragung der Studierenden befasst sich mit der aktuellen allgemeinen Situation, bezüglich der Bearbeitung der Bachelorarbeit.

\par\medskip Die Erkenntnisse lassen sich wie folgt zusammenfassen:
\begin{itemize}

\item \textbf{Erwartungen an den Betreuer}
\begin{itemize}

\item \textbf{Teilen von Erfahrungen}
\par Insbesondere werden sich Erkenntnisgewinne durch das Teilen von Erfahrungen der Betreuer gewünscht, was sich besonders auf die Arbeit mit externen Unternehmen bezieht. Hinweise zu Besonderheiten und die wichtigen Unterschiede zu einer internen Bachelorarbeit. 
\par Weiterhin das gemeinsame Erarbeiten eines Themas, was bezüglich des Umfangs durch die Erfahrungen des Betreuers eine angemessene Form annehmen sollte.

\item \textbf{Kommunikation}
\par Was die Kommunikationsbereitschaft betrifft, wird vor allem eine hohe Erreichbarkeit, das (schnelle) Beantworten von Fragen und die Möglichkeit, persönliche Treffen wahrnehmen zu können, von dem Betreuer erwartet.

\item \textbf{Feedback}
\par Grundsätzlich wird von dem Betreuer erwartet, dass dieser sich mit der Bachelorarbeit der Studierenden insofern beschäftigt, dass Feedbackpunkte zum aktuellen Stand, Inhalt, Ideen und Umfang konstruktiv diskutiert werden können. Darüber Hinaus erwarten die Studierenden, dass der Betreuer Hinweise bei Unverständlichkeiten der Doku gibt und gegebenenfalls schlechte Ideen und Entwurfsentscheidungen hinterfragt.
\par In diesem Rahmen wird auch vereinzelt erwartet, dass der Betreuer Interesse an der Arbeit des Studierenden hat, sowie Bereitschaft für die Beantwortung von Fragen und das Diskutieren von Lösungsmöglichkeiten eingeht.

\item \textbf{Impulse des Betreuers}
\par In wenigen Fällen erwarten die befragten Studierende, dass der Betreuer aktiv an dem Prozess der Erstellung der Bachelorarbeit teilnimmt. Das bezieht sich auf die Unterstützung bei dem Zeitmanagement, das Geben von Denkanstößen und Anregungen, sowie das von dem Betreuer ausgehenden Einfordern von Leistungen.
\end{itemize}

\item \textbf{Einschätzung der größten Probleme}

\begin{itemize}
\item \textbf{Schreiben der Dokumentation}
\par Die von den Studierenden angesprochenen Punkte befassen sich in erster Linie mit Unsicherheiten bei der Einschätzung von Umfang, Inhalt und Aufbau der Dokumentation. 
\par Des weiteren werden Rechtschreibung, sowie der korrekte Umgang mit Quellen genannt.

\item \textbf{Zeitmanagement}
\par Der Punkt Zeitmanagement stellt für viele Studierenden eine grundsätzliche die größte Herausforderung dar.

\item \textbf{Fachliche Probleme}
\par Teilweise werden auch fachliche Probleme unterschiedlicher Natur als größte Herausforderung genannt. Dazu zählen zum einen Probleme mit der im Rahmen der Bachelorarbeit zu verwendende Hardware, andererseits aber auch Softwareprobleme, das Entwickeln von sinnvollen Tests und das präzise Ermitteln von Anforderungen.

\item \textbf{Sonstige Probleme}
\par Weitere Probleme die vereinzelt genannt werden sind Probleme mit der Themenfindung, die Kommunikation zwischen internen Betreuer und der externen Firma oder die allgemeine Überforderung bei der Bearbeitung der Bachelorarbeit.
\end{itemize}

\item \textbf{Bessere Vorbereitung auf die Bachelorarbeit}
\begin{itemize}

\item \textbf{Zeitmanagement}
\par Im Allgemeinen wünscht sich ein großer Teil der Studierende eine bessere Vorbereitung bezüglich des Zeitmanagements. Die Studierenden bemängeln eine unzureichende Vorbereitung auf das Abschätzen von Zeitaufwand, Planung von größeren Projekten und nennen in diesem Rahmen den Wunsch nach Möglichkeiten und Maßnahmen im Studium, wie das häufigere Einbinden von Projektarbeiten.

\item \textbf{Wissenschaftliches Arbeiten}
\par Häufig wird der Wunsch nach besserer Vorbereitung auf das wissenschaftliche Arbeiten genannt, was sich explizit auf den Aufbau einer wissenschaftlichen Arbeit, der Umgang mit Quellen und das korrekte Einbinden von Quellcode-Inhalten. Als vorkommender Lösungsvorschlag wird ein Kurs, welcher zur besseren Vorbereitung vor dem 6. Semester stattfinden sollte, genannt.

\item \textbf{Aufklärung zur Bachelorarbeit}
\par Viele der befragten Studierenden wünschen sich vorbereitende und informierende Maßnahmen zur Bachelorarbeit. Dies bezieht sich vor allem auf das Klären organisatorischer Fragen, der wichtigen Formalien und der Umfang der Bachelorarbeit. Im Detail werden auch vorbereitende Maßnahmen gewünscht, die sich auf die Vor- und Nachteile einer extern oder intern betreuten Bachelorarbeit beziehen.

\item \textbf{Sonstige Anmerkungen}
\par Weiterhin treten bei der Befragung auch Fälle auf, bei denen keine expliziten Veränderung gewünscht werden, da die Studierenden zufrieden sind. Der Wunsch nach einem stärkeren Fokus auf Design-Pattern im Laufe des Studiums wurde einmal geäußert.

\end{itemize}

\end{itemize}

\newpage
\par \textbf{Einbezug der Studierenden in die Gestaltung der Applikation}
\par\medskip Die Einbindung der Studierenden befasst sich mit spezifischeren Fragen, hinsichtlich der motivierenden Gamificationinhalte der zu entwickelnden Applikation. Der Verlauf der folgenden Befragungen fand somit nicht in einem Durchlauf statt, sondern ergibt sich aus dem permanenten Einbinden der Studierenden. Es wurden bei der Befragung zwei Gamification-Ansätze vorgestellt.
\par\medskip Der erste Ansatz fällt unter anderem in die Benutzergruppe der \textbf{Socialiser} (Siehe Kapitel \ref{sec:grundlagenkapitelGamification}). Inhalt des Ansatzes ist es, die Motivation durch den sozialen Kontakt und Austausch mit anderen Teilnehmern zu erzeugen, indem diese den Fortschritt bei der Bearbeitung der Bachelorarbeit anderer Studierender wahrnehmen können.
\par\medskip Als Gegengewicht zum ersten Ansatz, orientiert sich der zweite Ansatz an der Benutzergruppe \textbf{Achievers} (Siehe Kapitel \ref{sec:grundlagenkapitelGamification}). Dieser Ansatz beruht vor allem auf das Abschließen und Sammeln von Erfolgen. Im Rahmen der Applikation wird es dem Benutzer bei dieser Idee ermöglicht, Erfolge durch erfüllen von definierten Aufgaben zu erlangen und somit den Fortschritt des Projektes durch das abschließen von geeigneten Aufgaben zu verfolgen.

\par\medskip Primäres Ziel des Einbezugs der Studierenden ist, zu bestimmen, welche Einbindung von Gamificationstrategien ansprechend auf die Zielgruppe der Studierenden wirken und in welcher Weise sich das Einbinden von Gamificationelementen auf die Steigerung der Motivation von Studenten bei der Bearbeitung der Bachelorarbeit auswirkt.

\par\medskip In der folgenden Tabelle werden die Ergebnisse der Befragung verdeutlicht:\\\\

\begin{tabularx}{\textwidth}{l|X|c}
	\hline
	\multirow{2}*{\textbf{Art der Gamificationstrategie}} & \textbf{Art der Reaktion} & 
	\textbf{Häufigkeit} \\ \hline
	\multirow{3}*{Motivation durch soziale Interaktion} & Positiv & 3 \\
	& Negativ & 12 \\
	& Neutral & 2 \\ \hline
	\multirow{3}*{Sammeln von Achievements und Erfüllen von Aufgaben} & Positiv & 15 \\
	& Negativ & 1 \\
	& Neutral & 1 \\ \hline
	&  & \textbf{17} \\
\end{tabularx}
\captionof{table}{Einbezug der Studierende in Wahl der Gamificationstrategie}
\label{tab:einbezugStudierende}

\newpage
\section{Untersuchung der Ergebnisse}

\subsection{Identifikation des Problembereichs}
\par Durch die Befragung der Beteiligten und der Analyse der gewonnen Informationen, haben sich die Erwartungen der Beteiligten an die Benutzung der Applikation herausarbeiten lassen, welche in der folgenden Tabelle (siehe Abbildung \ref{tab:erwartungenBeteiligte}) dargestellt sind.
\par Die dabei genannten Ziele und Erwartungen entsprechen dem allgemeinen Bild, welches durch die Interviews ermittelten werden konnte, weshalb kleine Abweichungen in dieser Tabelle nicht mit aufgenommen wurden. Kleinen Abweichungen hätten einen zu großen Einfluss auf das Gesamtbild. Die kompletten schriftlichen Befragungen und Interviews befinden sich im Anhang (TODO) dieses Dokuments.\\

\begin{tabularx}{\textwidth}{l|X}
	\toprule
	\textbf{Name der Beteiligten} & \textbf{Wünsche der Beteiligten} \\ \midrule
	{\textbf{Professoren}}
	& Steigerung der Fragebereitschaft von Studierenden\\
	& Weniger Aufwand in das sich wiederholende Informieren des Studierenden stecken\\ 
	& Zielgerichtetere Vorbereitung auf Besprechung mit dem Betreuer\\
	& Höhere Qualität von Abschlussarbeiten und der Dokumentation\\
	& Bessere Zeitplanung der Studierenden\\
	& Systematische(re) Vorgehensweise der Studierenden\\
	\multirow{6}*{\textbf{Studierende}} & Optimale Vorbereitung auf die Bearbeitung der Bachelorarbeit\\ \midrule
	& Optimales Ergebnis der Bachelorarbeit\\
	& Kenntnisse über Formalien, Umfang und Ablauf der Bachelorarbeit\\
	& Kontrolle über die Bachelorarbeit und den Ablauf\\
	& Unterstützung \\
	& Weniger Fallstricke bei der Bearbeitung der Bachelorarbeit stolpern\\ \bottomrule
\end{tabularx}
\captionof{table}{Erwartung der Beteiligten an die Applikation}
\label{tab:erwartungenBeteiligte}

\par Aus den Ergebnissen der Befragungen geht hervor, dass die Probleme nicht in den unterschiedlichen Ansichten der Studierenden und der Professoren liegen, da beide beteiligte Gruppen sehr kompatible Erwartungen und Ziele besitzen.

\par\medskip Häufig wird von den Studierenden geäußert, dass sie sich unzureichend vorbereitet für die Bachelorarbeit fühlen und die größten Probleme der Studierenden im Bereich Zeitmanagement und dem wissenschaftlichen Arbeiten liegen. Zusätzlich äußern sich die Studierenden zu fehlenden oder unzureichend aufgebauten Kenntnissen bezüglich der Rahmenbedingungen und weiteren Informationen zu Bachelorarbeiten als Ganzes.

\par\medskip Die Professoren sehen die Probleme bei den Studierenden eindeutig auch bei dem Zeitmanagement, dem Wissenschaftlichen Arbeiten, der Kommunikation und einer mangelnden Vorbereitung, bezüglich den Formalitäten und des Ablaufs einer Bachelorarbeit.

\par\medskip Hervorzuheben ist der Aspekt, dass die existierenden Probleme sowohl den Professoren, als auch den Studierenden bewusst sind und unabhängig der unterschiedlich ausfallenden Sichtweisen übereinstimmen.

\par\medskip Somit lässt sich erwarten, dass die angestrebten Ziele der Studierenden und die Erwartungen der Professoren aufgrund dieser Wissens- und Erfahrungslücken unter Umständen nicht reibungslos im Rahmen der Bachelorarbeit erfüllt werden können. Auf diese Weise werden die Erwartungen der Professoren, sowie die Ziele der Studierenden gefährdet und können unter Umständen nicht erfüllt werden, obwohl dies im Sinne beider Beteiligten ist.

\subsection{Auswahl der Motivationsstrategie}
\par Wie aus den Ergebnissen in Kapitel \ref{sub:studentenchriftlichErgebnisse} in Tabelle \ref{tab:einbezugStudierende} abzulesen ist, äußern sich die befragten Studierenden gegenüber dem vorgestellten Designvorschlag, welcher die soziale Einbindung der Studierenden vorsieht, überwiegend sehr kritisch. Im Vergleich dazu wird die vorgestellte Variante des Einsetzens von Achievements und Aufgaben in Form von Herausforderungen stark positiv reflektiert.
\par Aufgrund der eindeutigen Reaktionen der eingebundenen Studierenden, wird die Entscheidung getroffen, den ersten Ansatz nicht weiter zu verfolgen, dafür aber den zweiten Ansatz auszubauen. Somit fallen die Nutzergruppen überwiegend in den Spannungsbereich der \textbf{Explorers} und \textbf{Achievers}, wie in Abbildung \ref{img:interestGraphBartleAchiversExplorers} verdeutlicht wird. 

\begin{figure}[H]
	\centering
	\includegraphics[width=0.75\textwidth, keepaspectratio]{Bilder/Diagramme/InterestGraphBartleAchieversExplorers.png}
	\caption{Benutzeroberfläche Fortschrittsmanagement}
	\label{img:interestGraphBartleAchiversExplorers}
\end{figure}

\par Durch den vorgesehenen Kontext der Applikation, die Studierenden während der Bearbeitung der Bachelorarbeit wegweisend zu unterstützen und zu motivieren, bietet sich somit an, Herausforderungen und die damit verbundenen Achievements in Form von kontextbezogenen fokuslenkenden Game-Design-Elementen (siehe \ref{sec:grundlagenkapitelGamification}) in die Applikation als festen Bestandteil der Motivationsstrategie zu integrieren.

\par\medskip Wie in \citep{Sailer2016}[S. 32 - 34] beschrieben wird, besitzen Achievements verschiedene Eigenschaften, welche in Bezug auf Zielstellungen, einer steuernden und in diesem Sinne auch fordernden Funktion dienen können. Eine wichtiges Kriterium bei der Wahl der Achievements als Spiel-Design Elemente ist, dass sie dem Benutzer ein nicht-kontrollierendes, positives Feedback geben und somit nicht unangenehm gängeln oder den ohnehin schon sehr hoch ausfallenden Druck auf die Studierenden bei der Bearbeitung der Bachelorarbeit erhöhen. Es sollte in diesem Sinne nicht aus den Augen verloren werden, dass die Applikation den Studierenden positiv unterstützen soll.

\par\medskip Im weiteren Verlauf werden den Nutzergruppen entsprechend auf Spiel-Design-Elemente wie Bestenlisten oder anderen sozialen Motivationsmodellen verzichtet.

\par Weiterhin wird das Integrieren eines Punkte- oder Levelsystems, sowie einer Avatarfunktion nicht verfolgt. Dies dient der Absicht, dass die Applikation zwar motivierende Spiel-Design-Elemente enthalten soll, jedoch dies in einem Umfang passieren soll, der keine, an Gamification desinteressierten Benutzer, ausschließt und somit in diesem Fall eine weiterhin hilfreiche Lösung darstellen soll.

\newpage
\subsection{Abgeleitete Produktfunktionen und Anforderungen}
\par Im folgenden Verlauf sind die, aus der Aufgabenstellung und der Anforderungsanalyse erhobenen Produktanforderungen, in natürlicher Sprache vorzufinden. Diese wurden an einer Anforderungsschablone aus dem Lehrbuch der Softwaretechnik von Helmut Balzert orientiert und nach Bedarf spezifiziert\citep[Kapitel 19 - Natürlichsprachliche Anforderungen, Seite 481]{Balzert2010}:\\

\begin{enumerate}

\item \textbf{Unterstützung bei Zeit- und Fortschrittsmanagement}
\begin{enumerate}
\item Die Software muss dem Studierenden ermöglichen Meilensteine zu verwalten.
\begin{enumerate}
\item Das Anlegen von Meilensteinen
\item Das Entfernen von Meilensteinen
\item Das Ändern von Meilensteinen
\item Das Verschieben von Meilensteinen
\end{enumerate}

\item Die Software muss dem Studierenden ermöglichen, den Meilensteinen zugehörige Aufgabenpakete zu verwalten.
\begin{enumerate}
\item Das Anlegen von Arbeitspaketen
\item Das Entfernen von Arbeitspaketen
\item Das Ändern von Arbeitspaketen
\item Das Verschieben von Arbeitspaketen
\item Das Markieren von Arbeitspaketen als \textit{abgeschlossen}
\end{enumerate}

\item Die Software soll die Studierende auf bevorstehende Meilensteine und Aufgabenpakete aufmerksam machen.
\begin{enumerate}
\item Das Auslösen von Erinnerungen/Benachrichtigungen
\end{enumerate}
\end{enumerate}

\newpage
\item \textbf{Unterstützung bei der Bearbeitung der Bachelorarbeit}.
\begin{enumerate}
\item Die Software muss den Studierenden Informationen über die Rahmenbedingungen und Formalien der Bachelorarbeit an der Fachhochschule Lübeck bereitstellen können.
\item Die Software muss den Studierenden Hinweise und Empfehlungen zu Aufbau und Struktur von Bachelorarbeiten bereitstellen können.
\item Die Software muss den Studierenden Hinweise und Empfehlungen zur Durchführung von Bachelorarbeiten bereitstellen können.
\item Die Software muss den Studierenden eine Übersicht über studiengangs-spezifischen und -übergreifenden relevante Techniken und Methoden bereitstellen können.
\begin{enumerate}
\item Hilfestellung/Übersicht zu Themenfindung für Bachelorarbeit
\item Hilfestellung/Übersicht von studiengangs-spezifischen Inhalten
\item Hilfestellung/Übersicht zu Aufbereitung von Ergebnissen
\item Hilfestellung/Übersicht zu Nachweisführung
\end{enumerate}
\item Die Software muss den Umgang von zielgruppen-spezifischen Anpassungen der Inhalte ermöglichen können.
\item Die Anpassungen der Inhalte soll software-extern realisierbar und integrierbar sein.
\end{enumerate}

\item \textbf{Einsatz von Gamificationelementen zur Steigerung der Motivation}
\begin{enumerate}
\item Die Software muss im Rahmen der Gamificationstrategie, Spiel-Design-Elemente bereitstellen, welche der ermittelten Benutzerklassen entsprechen.
\item Die Spiel-Design-Elemente sollen unter Berücksichtigung der wegweisenden Eigenschaft, geeignet an dem Ablauf einer Bachelorarbeit orientiert sein.
\end{enumerate}

\item \textbf{Weitere Anforderungen}
\begin{enumerate}
\item Die Software muss sämtliche Inhalte und Leistungen auch für international-sprachige Personengruppen enthalten.
\item Die Software kann die Möglichkeit bieten, dass Betreuer ihre persönlichen Themenvorschläge für die Studierenden sichtbar, per Weboberfläche hinzufügen/entfernen können 
\end{enumerate}
\end{enumerate}

\chapter{Konzeptvorstellung der Applikation} \label{chap:konzept}

\section{Beschreibung der Software}
\par Die Applikation ist in verschiedene Kern-Softwareabschnitten unterteilt. Diese Abschnitte lassen sich in folgende namensgebende Aufgabenbereiche unterteilen, welche sich an den erhobenen Anforderungen und der Aufgabenstellung orientieren (siehe Kapitel \ref{sec:anforderungskatalog}).

\par\medskip Es folgt eine stichpunktartige Auflistung der Kernfeatures des Tools, welche im Laufe dieses Kapitels ausführlich beschrieben werden:

\begin{itemize}
\item \textbf{Fortschrittsmanagement}
\item \textbf{Guide}
\item \textbf{Dashboard}
\item \textbf{Herausforderungen}
\item \textbf{Achievements}
\item \textbf{Sonstige Softwareinhalte}

\end{itemize}

\newpage
\subsection{Fortschrittsmanagement}
\par Das Fortschrittsmanagement ist ein Tool, welches das Anlegen, Planen und Verwalten von Meilensteinen, sowie der Arbeitspakete ermöglichen soll. 

\par\bigskip Es folgt eine stichpunktartige Auflistung und Beschreibung der Kernfeatures des Tools:
\begin{itemize}
\item \textbf{Ein Zeitstrahl als Grundlage}
\par Der scrollbare Zeitstrahl bietet die Arbeitsgrundlage des Fortschrittsmanagement-Tools. Hier können wochenweise Meilensteine mit Arbeitspaketen angelegt werden.
\item \textbf{Erstellung und Verwaltung von Meilensteinen}
\par Es können Meilensteine erstellt, bearbeitet, verschoben und entfernt werden. Jede Woche kann maximal einen Meilenstein zugeordnet bekommen haben.
\item \textbf{Erstellung und Verwaltung von Arbeitspaketen}
\par Jeder Meilenstein enthält Arbeitspakete, die von dem Benutzer einem Meilenstein hinzugefügt, bearbeitet, verschoben und gelöscht werden können. Weiterhin lassen sich die Arbeitspakete optional konkreten Wochen zuordnen, um eine individuelle Planung zu ermöglichen.
\end{itemize}

\begin{figure}[H]
	\centering
	\includegraphics[width=0.4\textwidth, keepaspectratio]{Bilder/Prototyp/Zeitmanagement.png}
	\caption{Benutzeroberfläche Fortschrittsmanagement}
	\label{img:fortschrittsmanagement}
\end{figure}

\newpage
\subsection{Guide}
\par Der Guide stellt den Teil der Applikation dar, der die Bereitstellung von Hinweisen, Tipps und weiteren hilfreichen Informationen zur Bearbeitung der Bachelorarbeit abdecken soll. Dieser Inhalt ergibt sich aus der Verwendung des, von der Fachhochschule Lübeck ausgehändigten Dokuments, welches als Ratgeber bei der Erstellung von Bachelorarbeiten funktionieren und somit die speziellen Anforderungen der FH-Lübeck berücksichtigen soll \citep[vgl. Kapitel 1]{FHLuebeckBAAnleitung}.
\par In dieser Hinsicht soll der Guide als genereller Anlaufpunkt funktionieren, der beispielsweise eine Hilfestellung für Bacheloranden darstellen.

\par\medskip Die Bandbreite der Hinweise, Tipps und Informationen sollen somit den gesamten Verlauf der Bachelorarbeit abdecken und lassen sich in diesem Umfang in folgende Bereiche und Inhalte Zerlegen, welche sich an dem Leitfaden der FH-Lübeck\citep{FHLuebeckBAAnleitung} zur Erstellung einer Bachelorarbeit orientieren.

\par\medskip Es folgt eine eigene Interpretation und Beschreibung der Unterteilung der Themengebiete:

\begin{itemize}
\item \textbf{Allgemeine Informationen}
\par Dieser Abschnitt beinhaltet die allgemeinen Informationen, welche vor allem zur Klärung der Formalien bei der Bearbeitung der Abschlussarbeit wichtig sind.

\item \textbf{Struktur der Arbeit}
\par Der Abschnitt befasst sich näher mit dem Aufbau der Bachelorarbeit und stellt vor allem Informationen und Empfehlungen bereit, welche sich auf die Strukturierung und die Bedeutung der einzelnen Kapitel der Bachelorarbeit beziehen. 

\item \textbf{Hinweise zum Schreiben}
\par In diesem Abschnitt werden tiefgehende Informationen und Empfehlungen behandelt, welche sich auf die praktische Umsetzung des Schreibens im Detail beziehen. Beispielhaft hierfür sind der Schreibstil, die Verwendung von Zeiten oder die Einbindung von Bildern, Tabellen und Programmcode.

\item \textbf{Methodenübersicht}
\par Dieser Abschnitt bietet eine Übersicht, wichtiger Methoden der Informatik, die sich bei der Bearbeitung und Dokumentation der Bachelorarbeit als Hilfreich erweisen könnten.

\item \textbf{FAQ}
\par Dieses Kapitel behandelt oft gestellte Fragen und verdeutlicht die Informationen in der Form eines Frage-Antwort Schemas.
\end{itemize}

\begin{figure}[H]
	\subfigure[Guide Hauptbildschirm]{\includegraphics[width=0.33\textwidth]{Bilder/Prototyp/Guide.png}}
	\subfigure[Guide Themen-Ebene]{\includegraphics[width=0.33\textwidth]{Bilder/Prototyp/GuideThemenbereich.png}}
	\subfigure[Guide Informations-Ebene]{\includegraphics[width=0.33\textwidth]{Bilder/Prototyp/GuideThemenbereichInformationen.png}}
	\caption{Benutzeroberfläche Guide}
	\label{img:guide}
\end{figure}

\newpage
\subsection{Dashboard}
\par Das Dashboard stellt den Ausgangspunkt und Hauptbildschirm der Applikation dar und trägt somit die Aufgabe, aktuelle Informationen für den Nutzer aufbereitet anzuzeigen, sowie die Funktionen der Applikation auf einfache Weise zugänglich zu machen. Von diesem Punkt aus soll der Benutzer die verschiedenen Softwareabschnitte öffnen können, weshalb es eine essenzielle Eigenschaft des Dashboards ist, die Inhalte übersichtlich, strukturiert und gleichzeitig aber visuell ansprechend darzustellen.

\par\medskip Es folgt eine stichpunktartige Auflistung und Beschreibung der Inhalte:
\begin{itemize}
\item \textbf{Dynamische Anzeige des Fortschritts}
\par Der bisher geleistete Gesamtfortschritt wird verdeutlicht, indem zu sehen ist, wie viele Meilensteine in Relation zu den Gesamtmeilenstein bisher abschlossen wurden. Weiterhin wird der bisher geleistete Wochenfortschritt verdeutlicht, indem zu sehen ist, wie viele Aufgabenpakete in der Phase bis zum nächsten Meilenstein, in Relation zu den vorher definierten gesamten Arbeitspaketen, schon erfüllt wurden.
\item \textbf{Anzeige der noch offenen Herausforderungen}
\par Es wird angezeigt, wievielte Herausforderungen der jeweiligen Kategorien abgeschlossen wurden, in Relation zu deren Gesamtanzahl.
\item \textbf{Dynamische Anzeige der letzten erreichten Achievements}
\par Die Anzeige zeigt die letzten n erreichten Achievements an. Zugehörig sind in diesem Fall die Darstellung des Typs des Achievements in Form der jeweiligen Färbung, sowie den Titel der Herausforderung. Das zuletzt erreichte Achievement wird hierbei durch Größe der Darstellung und Beschreibung hervorgehoben.
\item \textbf{Dynamische Anzeige von allgemeinen Tipps}
\par Die Anzeige zeigt zufällig ausgewählte allgemeine Tipps zur Bearbeitung der Bachelorarbeit. Die Tipps wechseln nach einem definierten Zeitintervall.
\end{itemize}

\begin{figure}[H]
	\centering
	\includegraphics[width=0.45\textwidth,keepaspectratio]{Bilder/Prototyp/Dashboard.png}
	\caption{Benutzeroberfläche Dashboard}
	\label{img:dashboard}
\end{figure}

\newpage
\subsection{Herausforderungen}
In diesem Bereich der Applikation kann der Benutzer einsehen, welche Herausforderungen schon abgeschlossen wurden und welche noch offen sind. Die Herausforderungen sind nach festgelegten Kategorien geordnet. Bei öffnen der Kategorien, gelangt der Benutzer in eine tiefere Ebene, in der die jeweiligen, noch zu erledigen oder abgeschlossenen Herausforderungen, einzusehen sind.

\begin{figure}[H]
	\centering
	\includegraphics[width=0.4\textwidth, keepaspectratio]{Bilder/Prototyp/AufgabenSortiert.jpg}
	\caption{Benutzeroberfläche Herausforderungen}
	\label{img:aufgaben}
\end{figure}

\newpage
\subsection{Achievements}
\par Der Achievement Hauptbildschirm enthält eine Übersicht der verschiedenen Kategorien von Achievements (siehe Abbildung \ref{img:achievements}). Die nächste Ebene dieses Abschnitts zeigt die detaillierte Ansicht der jeweiligen Achievements an, wo sich der Benutzer einsehen kann, welche Achievements er bereits erhalten hat. Weitere Informationen zu den einzelnen Achievement und deren Kategorien finden ich im Kapitel \ref{sec:gamificationelemente}\\

\begin{figure}[H]
\centering
	\subfigure[Achievements Hauptbildschirm]{\includegraphics[width=0.35\textwidth]{Bilder/Prototyp/AchievementsUebersicht}}
	\subfigure[Achievements]{\includegraphics[width=0.35\textwidth]{Bilder/Prototyp/AchievementsUebersichtAchievements.jpg}}
	\caption{Benutzeroberfläche Achievements}
	\label{img:achievements}
\end{figure}

\newpage
\subsection{Sonstige Softwareinhalte}
\par Über die eigentlichen individuellen Funktionalitäten hinaus, sind weitere Grundfunktionalitäten vorhanden, welche unter den Punkt \textbf{Sonstige Softwareinhalte} fallen.

\par\medskip Es folgt eine stichpunktartige Auflistung und Beschreibung der Inhalte:
\begin{itemize}
\item \textbf{Menü}
\par Das Menü zeigt die existierenden Funktionalitäten und Softwareabschnitte in einer klassischen gelisteten Menüstruktur.
\item \textbf{Einstellungen}
\par Die Einstellungen beinhalten Optionen, die sich auf die Eigenschaften der Applikation auswirken. Beispielsweise das Verändern des Designs.
\end{itemize}

\begin{figure}[H]
	\centering
	\includegraphics[width=0.4\textwidth,keepaspectratio]{Bilder/Prototyp/Menu.png}
	\caption{Benutzeroberfläche Navigations-Menü}
	\label{img:navigation}
\end{figure}

\newpage
\section{Beschreibung der Gamification-Elemente} \label{sec:gamificationelemente}
\par Im folgenden Kapitel werden die Gamification-Elemente, welche in der Applikation zum Einsatz kommen, näher beschrieben. Dabei wird vor allem auf die identifizierten Kategorien der Achievements eingegangen, welche den Kern der gewählten Motivationsstrategie bilden. Darüber Hinaus wird auch beschrieben, in welcher Form die gewählte Motivationsstrategie in der Realität, in Bezug auf den Fortschritt bei der Bearbeitung der bachelorarbeit, funktionieren soll.

\subsection{Anfangsphase Achievements}

\subsection{Bearbeitungsphase Achievements}
\subsection{Abschlussphase Achievements}
\subsection{App-Achievements Achievements}
\subsection{Fortschrittsverfolgung}
- Weekly Tasks
- Abschlussbericht pro Woche?

\chapter{Architektur der Software} \label{chap:architektur}
\section{Architekturbeschreibung}
\section{Entwurfsentscheidungen}
\section{Erweiterbarkeit der Software für andere Studiengänge}
\chapter{Implementierung} \label{chap:implementierung}
\section{Umsetzung der Anforderungen}
\section{Entwurf der Benutzeroberfläche}

\chapter{Validierung und Verifikation} \label{chap:nachweisführung}
\section{Softwaretests}
\section{Abdeckung der Softwareanforderungen}
\section{Ausführung der Usabilitytests}

\chapter{Präsentation der Ergebnisse} \label{chap:ergebnisse}
\section{Lessons Learned}
\section{Abschlussbetrachtung}

\nocite{*}
\printbibliography
\end{document}
